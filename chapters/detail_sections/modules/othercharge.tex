\subsection{Other Charge Module}

\subsubsection{Task}
\begin{itemize}
    \setlength\itemsep{1em}
    \sloppy
    \item \textbf{{GISLOCK-5943 [FA-Other charge request] จัดทำหน้าจอ Other charge request}} \\
          เป็นการเพิ่มโปรแกรมใหม่ในแผนกของการเงิน โดยการเป็นสร้างโปรแกรมที่ชื่อว่า Other Charge Request และ Other Charge Request List\\
          Other Change Request : เป็นโปรแกรมเกี่ยวกับการเรียกค่าใช้จ่ายเพิ่มเติมนอกเหนือจากเบี้ยประกัน \\
          Other Change Request List : เป็นหน้าที่เอาไว้รายการที่เราสร้างไว้
    \item \textbf{{GISLOCK-6755 [Set up Transaction Code] เพิ่มหน้าจอ Set up Transaction Code}} \\
          เป็นการเพิ่มโปรแกรมใหม่ที่เราจะเแาไว้ตั้งค่ารายการ Transaction Code ที่จะใช้ในโปรแกรม Other Charge Request กับ Other Expense Request \\
          โดยจะแบ่ง 2 โปรแกรมคือ Set up Transaction Code และ Set up Transaction Code List 
    \item \textbf{{GISLOCK-7019 [Other Charge] เพิ่มปุ่ม Update (สำหรับ Refresh) ข้อมูล Installment และ ShareDept}} \\
          เป็นการเพิ่มปุ่ม Update ในหน้าจอ Other Charge Request เพื่อให้ผู้ใช้สามารถกดปุ่มนี้เพื่อรีเฟรชข้อมูลในตาราง Installment และ Share Department ได้ แบบอัตโนมัติ
    \item \textbf{{GISLOCK-7106 [Other Charge] แก้ไขโปรแกรม Other charge Request List เพิ่มเงื่อนไขในการ query เพื่อค้นหาข้อมูลโดยกรองเฉพาะ Entry\_by ที่ตรงกับ User Login}} \\
          เป็นการแก้ไขโปรแกรม Other Charge Request List โดยเพิ่มเงื่อนไขในการค้นหาข้อมูลให้สามารถกรองเฉพาะ Entry\_by ที่ตรงกับ User Login นั้น ๆ ได้ \\
          อนาคตจะแบ่งให้แสดงแค่แผนกที่ตัวเองรับผิดชอบ
    \item \textbf{{GISLOCK-7200 [Other Charge] แก้ไขโปรแกรม Charge Request ส่วนของการ Share dept \, Installment}} \\
          ปรับการแสดงข้อความเตือน เมื่อมีการกรอกข้อมูลไม่ครบถ้วนในส่วนของ Share Department และ Installment \\
          โดยจะมีการเช็คให้ข้อมูลของ Installment และ Share Department ต้องมีค่าเปอร์เซ็นต์รวมกันเท่ากับ 100\% ถึงจะสามารถบันทึกข้อมูลได้ \\
          และจะต้องมีจำนวนเงินที่สัมพันต์กับจำนวนเงินใน Transaction ด้วย
    \item \textbf{{GISLOCK-7289 [Other Charge] Set up Transaction Code เพิ่ม DropdownFlag}} \\
          เป็นการเพิ่มตัวเลือก DropdownFlag ในหน้าจอ Set up Transaction Code \\
          เพื่อให้ผู้ใช้สามารถเลือกได้ว่ารายการ Transaction Code ที่สร้างขึ้นมานั้น จะนำไปแสดงในหน้าจอ Other Charge Request หรือไม่
    \item \textbf{{GISLOCK-7688 [Other Charge] Other Charge Request Booking}} \\
          เป็นการเพิ่มโปรแกรมในส่วนของแผนกการเงิน เป็นโปรแกรมที่แผนกการเงินจะเอาไว้ตรวจสอบหรือแก้ไขรายการที่ถูกอนุมัติแล้ว \\
          โดยจะมีการแยกเป็น 2 โปรแกรมคือ Other Charge Booking และ Other Charge Booking List
    \item \textbf{{GISLOCK-7689 [Other Charge] Other Charge Request บันทึกขาคู่}} \\
          เป็นส่วนหนึ่งของโปรแกรม Other Charge Request ที่เพิ่มขึ้นมา \\
          โดยจะเป็นการบันทึกข้อมูลแบบขาคู่ คือการที่เราจะต้องบันทึกข้อมูลทั้งในฝั่งของ Other Charge Request และ Other Expense Request พร้อมกัน \\
          โดยจะมี 2 แบบ คือ : \\
            1. แบบ auto ที่จะสร้าง Other Expense Request ให้อัตโนมัติเมื่อมีการบันทึก Other Charge Request \\
            2. แบบ manual ที่จะให้ผู้ใช้เป็นคนเลือกว่าจะสร้าง Other Expense Request ด้วยตัวเอง
\end{itemize}

\subsubsection{Defect}
\begin{itemize}
    \setlength\itemsep{1em}
    \sloppy
    \item \textbf{{GISLOCK-5457 [Other Charge] ปรับการแสดงผลข้อมูลในส่วน Bill to ให้ไปในทางเดียวกับ AP}} \\
          เพิ่มการ Clear โดยการกดปุ่ม X ที่ Bill To
    \item \textbf{{GISLOCK-6189 [Other Chare/Expense] Standard Search Search like เวลาเก็บ Data ใน DB ให้คั่นด้วย Space Bar}} \\
          1.ปรับตรง placeholder ของการ Search เราจะเอาการเว้นวรรคออก จะแสดงชิดกันเลย โดยมี / คั่น
          2.ตรง Database จะปรับจากการเก็บที่คั่นโดย | มาใช้เป็น " " (Spacebar หรือ การเคาะช่องว่างแทน) 
    \item \textbf{{GISLOCK-6518 [FA- Other charge request] ปรับ feature ทำหน้าจอ Other charge request}} \\
          เป็นการแก้หลายๆ จุดในโปรแกรม Other Charge Request ที่ได้มีการพัฒนาขึ้นมา \\
            โดยมีการปรับปรุงดังนี้ \\
            1. ปรับปรุงการแสดง ปุ่ม Add/Clear ของส่วน Share Department ให้ตำแหน่งปรับตำขนาดหน้าจอ\\
            2. ปรับปรุงหน้า Approve เมื่อกด view mode มาต้องไม่มีปุ่ม Add/Clear ของแต่ละส่วนแสดง \\
            3. view mode ที่กดมาจากหน้า List ส่วนของ Transaction จะต้องเอาข้อมูลของชุดแรกมาแสดง \\
            4. Hq/Branch แสดงเป็นข้อมูล ไม่ต้องแสดงโค้ด \\
            5. Exchange Rate ใน Transaction ให้แสดง 8 ตำแหน่ง \\
            6. เมื่อเราเพิ่ม Transaction แล้วเราจะทำการปิด Currency/Exchange Rate ไม่ให้แก้ไขได้ เพื่อ Transaction ต่อไปจะได้ Currency/Exchange Rate เหมือนกัน\\
            7. กรณีที่เรามี Installment หรือ Share Department แล้วเราไปแก้ไข Transaction เมื่อจะเซฟเราจะต้องเตือนบอกว่า Transaction มันเปลี่ยนแปลง และเซฟไม่ได้ 
    \item \textbf{{GISLOCK-6561 [Other Charge] Pop up \: Bill To แสดงข้อมูลเดียวกัน ซ้ำๆ}} \\
          สาเหตุที่ข้อมูลแสดงซ้ำ เนื่องจากมีการเชื่อมโยง (JOIN) กับตารางข้อมูลอื่น เช่น ตารางที่อยู่ (Address) ซึ่งมีหลายรายการต่อรหัสลูกค้าหรือพาร์ทเนอร์ ทำให้ข้อมูลหลักหนึ่งรายการถูกแสดงซ้ำตามจำนวนข้อมูลที่อยู่ที่สัมพันธ์กันในตารางดังกล่าว \\
          วิธีแก้ไขคือ การปรับคำสั่ง SQL ที่ใช้ในการดึงข้อมูล โดยการใช้คำสั่ง DISTINCT เพื่อให้แสดงเฉพาะรายการที่ไม่ซ้ำกัน หรือการปรับโครงสร้างการ JOIN ให้เหมาะสมกับความต้องการของข้อมูลที่จะแสดง
    \item \textbf{{GISLOCK-6563 [Other Charge] Contact Information จะต้องดึงข้อมูลมาจาก Contact Person ของ Client หรือ Partner มาแสดงโดย link up กับ Bill to ที่ได้เลือก}} \\
          ปรับปรุงตรง Contact Information :\\
          1.ดึงข้อมูลมาแสดงอัตโนมัติตาม Main Contact \\
          2.กรณีมีหลาย Contact เปลี่ยนได้กดเปลี่ยนตรง modal จะมี Contact ที่สามารถเปลี่ยนได้
    \item \textbf{{GISLOCK-6566 [Other Charge] ปรับ wording ของชื่อฟิลด์ ทั้งหน้า Request และ List}} \\
          เป็นการแก้ไขคำของหน้า Request และ List
    \item \textbf{{GISLOCK-6568 [Other Charge] ตรวจสอบการ Save Draft ใบงานและการแสดงค่าเริ่มต้นฟิลด์ Status}} \\
          เพิ่มการเช็คตอนจะ Save Draft ใบคำขอ ให้เช็คว่าเราเลือกตัวที่ต้อง require หรือยัง ถ้ายังให้แจ้งเตือน \\
          และปรับการแสดงค่าเริ่มต้นของฟิลด์ Status ให้แสดงเป็น "Save Draft" เมื่อสร้างใบคำขอใหม่
    \item \textbf{{GISLOCK-6571 [Other Charge] ส่วน Share Department ปรับการแสดงปุ่ม Add \, Clear และเมื่อเลือกข้อมูล Department รูปแบบฟอร์มต้องไม่ขยับไปมา}} \\
          ปรับปรุงการแสดงปุ่ม Add และ Clear ในส่วนของ Share Department ให้มีตำแหน่งที่คงที่ ไม่ขยับไปมาเมื่อมีการเลือกข้อมูล 
    \item \textbf{{GISLOCK-6575 [Other Charge] การใช้งานปุ่ม View ในส่วน Other Charge Transaction}} \\
          เพิ่มปุ่ม Back ในหน้าจอ View ของส่วน Other Charge Transaction \\
          เพื่อให้ผู้ใช้สามารถกลับไปยังหน้าจอก่อนหน้าได้อย่างสะดวก
    \item \textbf{{GISLOCK-6576 [Other Charge] กรณี Total Amount ใน Transaction และ Installment มียอดไม่เท่ากัน เมื่อ Save \, Submit จะต้องมี Pop up ถาม}} \\
          เมื่อ Total Amount ในส่วนของ Transaction และ Installment มียอดไม่เท่ากัน \\
          ระบบจะต้องแสดง Pop-up แจ้งเตือนผู้ใช้ว่า ยอดรวมไม่ตรงกัน และให้ผู้ใช้ทำการตรวจสอบและแก้ไขข้อมูลก่อนที่จะทำการ Save หรือ Submit ใบคำขอ
    \item \textbf{{GISLOCK-6577 [Other Charge List] ปรับการใช้งาน Request Date From / To}} \\
          1.เพิ่มตรง Coondition Search โดยการเพิ่มการค้นหาด้วย Status 
          2.ไม่ Default วันที่ให้ตอนเริ่มเพราะให้มาเลือกเอง
    \item \textbf{{GISLOCK-6578 [Other Charge List] ฟิลด์ Bill To กดแล้วไม่แสดง Pop up และขอความหมายของ Request No. Running No. C2568XXXXX}} \\
          สาเหตุที่กดแล้วไม่แสดง Pop-up เกิดจากการที่เราไปเรียก id ของ Pop-up นั้นไม่ถูกต้อง
    \item \textbf{{GISLOCK-6583 [Other Charge List] ไม่แสดงข้อมูล Bill to Name}} \\
          เกิดจากการที่ก่อนหน้านี้มีการเก็บข้อมูล Bill to Name ในตารางที่ไม่ถูกต้อง \\
          วิธีแก้ไขคือ ตอนที่เราจะเซฟใบงานเราจะเลือกให้ถูกฟิลด์พอตอนจะเรียกมาหน้า List ก็จะได้ข้อมูลที่ถูกต้อง
    \item \textbf{{GISLOCK-6589 [Other Charge List] การใช้งานและความหมายของ Submit Date}} \\
          เพิ่มคำอธิบายใต้ฟิลด์ Submit Date ในหน้า Other Charge List 
    \item \textbf{{GISLOCK-6591 [Other Charge] ปรับชื่อฟิลด์ Entry By และ Entry Date ให้กับหน้าจอ Request และ List}} \\
          เปลี่ยนมาใช้ชื่อฟิลด์เป็น Entry By และ Entry Date ในหน้าจอ Request และ List 
    \item \textbf{{GISLOCK-6592 [Other Charge List] เพิ่มการแสกงข้อมูลในคอลัมน์ Invoice Date}} \\
          เพิ่มการแสดงข้อมูลในคอลัมน์ Invoice Date ในหน้า Other Charge List 
    \item \textbf{{GISLOCK-6649 [Other Charge][Request Other Charge] ปรับการทำงานส่วน Contact Information ให้มีฟังก์ชันการทำงานเหมือนกับ AP}} \\
          ปรับการทำงานตามนี้ \: \\
          1. เมื่อเลือก Bill to Code แล้วให้ดึงข้อมูล Contact Information มาแสดงอัตโนมัติ โดยเลือกจาก All Purpose เป็นลำดับแรก \\
          2. หากมีหลาย contact สามารถกดปุ่มแว่นขยายเพื่อแสดง modal ของ contact ที่มีหลายอันแล้งเลือกได้
    \item \textbf{{GISLOCK-6675 [Other Charge][Request Other Charge] ส่วน Request Information ทำการ Edit ข้อมูลในตารางแต่ระบบไม่ Update ตาม}} \\
        เกิดจากการที่เราส่งข้อมูลไปแต่ว่าไอดีในการอัปเดตข้อมูลนั้นไม่ถูกส่งไป เลยไม่รู้ว่าจะแก้ไขที่ข้อมูลไหน ทำการแก้ไขโดยการส่งไอดีเพิ่มไปด้วย
    \item \textbf{{GISLOCK-6773 [Other Charge][Request Other Charge] Submit ใบงาน ระบบแสดง The sum of all tables is not equal.}} \\
        เพิ่มการแสดง Modal เมื่อผลรวมของ Total Amount ไม่เท่ากันใน Installment และ Share Department
    \item \textbf{{GISLOCK-6877 [OTC][Request Other Charge] กรณีเลือก Edit รายการเพื่อตรวจสอบข้อมูล แต่ Entry Date แสดงเป็นวันที่ปัจจุบัน ซึ่งไม่ถูกต้องให้แก้ไขเป็นวันที่บันทึกใบงานตั้งต้น}} \\
        จากเดิมก่อนหน้านี้เป็นการ mock up ข้อมูลมาเลยทำให้การแสดงเป็นวันปุจจุบันเสมอ เราจะแก้ไขใน mode edit ให้ไปดึงข้อมูลจากฐานข้อมูลมาแสดงแทน
    \item \textbf{{GISLOCK-6878 [OTC][Request Other Charge] กรณีเลือก Invoice Name ฟิลด์ Invoice Address จะต้องแสดงข้อมูลที่สัมพันธ์กับ Name ที่เลือก}} \\
        ปรับปรุงการทำงานของฟิลด์ Invoice Address ให้แสดงข้อมูลที่สัมพันธ์กับ Invoice Name ที่ผู้ใช้เลือก \\
        โดยการเชื่อมโยงข้อมูลจากฐานข้อมูลเพื่อดึงที่อยู่ที่ตรงกับชื่อที่เลือกมาแสดง
    \item \textbf{{GISLOCK-6945 [Other Charge List] ปรับการแสดง Location และ Nationality กรณี Bill To เป็นข้อมูลเก่า โดยระบบต้องแสดงข้อมูลที่ได้เลือกจากตอนบันทึก}} \\
        เดิมทีเรามีการเก็บ เป็นคำ location ประมาณนี้แต่เรามีการปรับรูปแบบใหม่ให้เป็นเป็นคีย์ L แบบนี้แทน เราเลยมีเขียน sql เพื่อดึงข้อมูลเดิมมาแล้วดักเคสที่ส่งออกไป ตอนเราจะดึงข้อมูลมาแสดงก็จะได้ข้อมูลใหม่ที่เป้น คีย์ L แทน\\
        แล้วตอนเซฟเราก็จะเซฟเป็นคีย์ L ไปเลย ทำให้เป็นการอัปเดตข้อมูลให้เป็นรูปแบบใหม่ทั้งหมด
    \item \textbf{{GISLOCK-6953 [Other Charge] 1.แก้ไขการแสดง Request Information 2.แก้ไขการแสดง Currency/Exchange Rate 3.แก้ไขการแสดง Bill to Code}} \\
        1. เมื่อทำการกด Icon Edit ข้อมูล Transaction Code ในตารางและไปกดปุ่ม Request Information ข้อมูลเดิมจะต้องไม่หาย \\
        2. เมื่อ Edit เข้ามาจากหน้า List ถ้ามี Transaction อยู่เราจะ Disable Currency และ Exchange Rate เพราะจะใช้ตามอันที่มี
        3. เมื่อ x กากบาท Code หรือ x กากบาท Pop up จะต้องเรียก Service ใหม่ทุกครั้ง เป็นการ refresh ทุกครั้งที่เรากดมาใหม่
    \item \textbf{{GISLOCK-7031 [Other Charge] ปรับการทำงานปุ่ม Submit และ Save Draft}} \\
        จากเดิมเมื่อเป็น view mode ปุ่ม Submit กับ Save Draft มันจะ Disable แต่ยังกดและทำงานได้ \\
        เราเลยปรับให้ปุ่มมันกดไม่ได้เลยเมื่อเป็น view mode 
    \item \textbf{{GISLOCK-7064 [Other Charge] เมื่อ Submit and Save Draft ข้อมูลที่บันทึกไว้ในหน้าจอต้องไม่เคลียร์และ Redirect to List}} \\
        จากอันเดิมก่อนหน้านี้พอเรากด Submit หรือ Save Draft ข้อมูลมันจะเคลียร์ออกจากหน้าจอและนำทางไปสู่หน้าลิสต์ \\
        เราเลยปรับให้เมื่อกด Submit หรือ Save Draft ข้อมูลที่บันทึกไว้ยังคงอยู่ในหน้าจอ และจะ Redirect to List เฉพาะเมื่อผู้ใช้เลือกที่จะกลับไปยังหน้าลิสต์ด้วยตัวเอง
    \item \textbf{{GISLOCK-7144 [Other Charge List] ยอด Amount แสดงไม่ตรงกับการบันทึกในหน้า Request}} \\
        เปลี่ยนจากเดิมเราจะใช้ผลรวมของ Amount ในตาราง Transaction มาแสดง \\
        เป็นการไปดึงยอดผลรวมของ Total Amount มาแสดงแทน
    \item \textbf{{GISLOCK-7215 [Other Charge] ตรวจสอบการคำนวณ Payment Due Date ให้ตรงตามเงื่อนไข}} \\
        ปรับปรุงการคำนวณ Payment Due Date ในหน้าจอ Other Charge Request \\
        โดยจะมีเงื่อนไขดังนี้ \\
        1. การคิด Payment Due Date จะเป็น วันที่สร้างใบคำขอ + payment term ของคนนั้นๆ แต่ถ้าไม่มีจะเป็นค่าเริ่มต้นให้ 30 วัน\\
        2. ถ้า Payment Due Date ตรงกับเสาร์อาทิตย์ และ วันหยุดนักขัตฤกษ์ จะเลื่อนไปเป็นวันทำการถัดไปโดยอัตโนมัติ
    \item \textbf{{GISLOCK-7219 [Other Charge] เพิ่มการแจ้งเตือน Share Department และ Installment กรณีที่ Share (\%) มีจำนวน 0.00 และมากกว่า 100.00}} \\
        ปรับการแจ้งเตือนจากเดิมไม่มีจุดทศนิยม \\
        เป็นการเพิ่มจุดทศนิยม 2 ตำแหน่งในการแจ้งเตือนเมื่อ Share (\%) มีจำนวน 0.00 และมากกว่า 100.00 \\
        เพิ่มการดักเคสที่กรอกค่าเกิน 100.00 และน้อยกว่า 0.00 หรือเท่ากับ 0.00
    \item \textbf{{GISLOCK-7230 [Other Charge] แก้ไขการกดปุ่ม Clear และ เพิ่ม Popup has been saved ให้กับการเพิ่มข้อมูลลงตาราง}} \\
        เพิ่มการแสดง modal เมื่อ add update และ delete เพื่อบอกว่าทำเสร็จแล้วของ Transaction , Request Information , Share Department , Installment
    \item \textbf{{GISLOCK-7241 [Other Charge] ลบรายการ Transaction ออกไปแล้ว ข้อมูลส่วน Installment และ ปุ่ม Update ยังแสดงอยู่ในหน้าจอ}} \\
        เพิ่มการเช็คตอนที่เราจะเซ็ตค่าใหม่ให้อัปเดตตามเงื่อนไข
    \item \textbf{{GISLOCK-7243 [Other Charge] กดปุ่ม Update แล้วไม่ Update ข้อมูล กรณี Installment และ Share Dept มีข้อมูลทั้งคู่}} \\
        เกิดจากการที่เราส่ง payload ไปให้หลังบ้านไม่ถูกเลยทำให้การอัปเดตทำงานไม่ถูกต้อง \\
        แก้ไขโดยการปรับ payload ก็จะสามารถอัปเดตได้ปกติ
    \item \textbf{{GISLOCK-7258 [Other Charge] เพิ่มการปรับ \% ของ Share Department และ Installment}} \\
        เพิ่มการคำนวณเปอร์เซ็นต์ให้อัตโนมัติเช่น เริ่มต้นเรามี 100 แล้วเราเพิ่มอันแรกไป 20 แล้วมันก็จะเหลือ 80 ให้ใช้ครั้งต่อไป รวมถึงการคำนวณ total amount ด้วย
    \item \textbf{{GISLOCK-7395 [Other Charge] เมื่อเลือก Vat Calculation = Non Vat ให้ Disable Field 1.Vat Type 2.Vat Rate \% 3.Vat Amount}} \\
        จะทำการปิดฟิลด์ดังกล่าวเมื่อเลือก Vat Calculation เป็น Non Vat
    \item \textbf{{GISLOCK-7419 [Other Charge] ส่วน Contact Information เพิ่มเงื่อนไขเมื่อไม่มีการเลือก Code ให้ Disable ส่วน Contact ไว้ก่อนเสมอ}} \\
        ปรับปรุงให้เมื่อเราเข้ามาครั้งแรกตอนที่เรายังไม่เลือก ลูกค้าหรือพาร์ทเนอร์ เราจะ Disable ส่วนของ Contact ไว้ก่อน
    \item \textbf{{GISLOCK-7439 [Other Charge] Vat Rate \% = 0 ให้ Disable Vat Amount}} \\
        ใน Transaction เราเราเลือก Vat Rate เป็น 0 gik0tmedki Disable Vat Amount ไปเลย เพื่อจะได้ไม่สามารถแก้ไขได้เพราะเราเลือก 0 แล้ว 
    \item \textbf{{GISLOCK-7487 [Other Charge] ตรวจสอบการค้นหาข้อมูลแบบ ค้นหามีช่องว่างระหว่างคำ}} \\
        ปรับรูปแบบการ search ใหม่ จากเดิมเราจะ search ได้ทีละตัว แต่เราจะทำการเพิ่มไป เป็นการ search แบบหลายตัว เช่น "AA BA" \\
        เราจะแสดงผลลัพธ์ที่ตรงกับ AA มาทั้งหมดก่อน ต่ามด้วย BA ต่อด้วย
    \item \textbf{{GISLOCK-7494 [Other Charge] ปรับตำแหน่งการแสดงปุ่ม Search และ Clear}} \\
        ปรับตำแหน่งการแสดงปุ่ม
    \item \textbf{{GISLOCK-7498 [Other Charge] Set up Transaction Code แก้ไขตามรายการ}} \\
        1. Account Code ทำเป็น * ฟิลด์เพื่อป้องปันการ Add ลงตารางโดยที่ไม่เลือกข้อมูล \\
        2. เพิ่มการทำงานของปุ่ม Clear หลัก \\
        3. ปรับ Icon : edit, delete ให้ตรงกับโปรแกรม Standard \\
        4. Account Code จะต้องแสดงแด่ Code และ Account Name สามารถดูได้อย่างเดียวไม่สามารถแก้ไขได้ \\
        7. ปรับฟิลด์ Type มาเป็น Transaction Type ตามหน้า List
        8. เพิ่มการ Validate WHT Rate * และเป็น require field
    \item \textbf{{GISLOCK-7529 [Other Charge] โปรแกรม Set up Transaction Code ทำการ Delete รายการในตารางแล้ว รายการไม่ถูกลบไป}} \\
        ปัญหาเกิดจาก Race Condition คือคำสั่ง GET ทำงานก่อนที่คำสั่ง DELETE จะเสร็จสมบูรณ์ ทำให้เห็นข้อมูลที่ยังไม่ถูกลบจริง \\
        รอให้การลบเสร็จสิ้นก่อนดึงข้อมูล หรือใช้ Transaction/Locking เพื่อควบคุมลำดับการทำงานไม่ให้คำสั่งชนกัน.
    \item \textbf{{GISLOCK-7554 [Other Charge] ลบรายการที่มี Vat Amount ที่ส่วน Transaction ออกไปแล้ว แต่ในส่วน Installment ยังค้างยอด Vat Amount เก่าไว้}} \\
        เกิดจากการที่เราทำการแก้ไขใน Transaction แล้วในส่วนของ Installment ไม่ได้รู้ว่ามันเปลี่ยน เลยแก้ไขโดยการสร้างฟังก์ชันที่จะเช็คว่าถ้ามีการเปลี่ยนแปลงและเข้าเงื่อนไขที่เราตั้งก็จะไปทำฟังก์ชันนี้ก็จะสามารถทำงานได้ปกติ
    \item \textbf{{GISLOCK-7587 [Other Charge] Set up Transaction Code เปลี่ยนจากปุ่ม Submit/Clear เป็น Save/Clear}} \\
        เปลี่ยนหน้า Ui ของปุ่ม
    \item \textbf{{GISLOCK-7626 [Other Charge] เปลี่ยนจากการส่งข้อมูลแบบ queryParams เป็นแบบ getState}} \\
        ปรับจากเดิมที่เป็นการรับมาจาก Params เราปรับเปลี่ยนมาใช้ในรูปแบบ State ซึ่งปลอดภัยและใช้งานได้ดีกว่า
    \item \textbf{{GISLOCK-7628 [Other Charge] ปรับ status เป็นตัวกลางเพื่อให้ sort จะได้ เนื่องจากตอนนี้เป็นการใช้เงื่อนไขในการแปลงจาก code เป็น word และนำมาแสดง ไม่ใช่การ Join}} \\
        เพิ่ม status ของ Other Charge ให้ไปเป็น Workflow ของตัวกลางเพื่อที่จะได้เข้ากระบวนการ approval ของส่วนกลาง และ จะสามารถนำมา Sort ได้
    \item \textbf{{GISLOCK-7629 [Other Charge] ลบข้อมูลในตาราง Installment แล้วในตาราง Transaction มีแต่เคส VATNO ระบบไม่ Disable Field Vat ในส่วน Installment ให้}} \\
        เกิดจากตอนที่เรา แก้ไขข้อมูลใน Installment แล้วเราไม่ได้เช็คของ Transaction เลยทำให้ตอนเราลบแล้วจะเซ็ตใหม่เลยกลับมาเปิด Vat Amount ทั้งๆที่ใน Transaction ยังมี VATNO \\
        เราแก้ไขโดยการเพิ่มการเช็คไปตอนเราเคลียร์เพื่อจะเซ็ตใหม่ ก็จะทำให้ฟิลด์เปิดปิดตามข้อมูล Transaction ที่เรามี
    \item \textbf{{GISLOCK-7631 [Other Charge][Set up Transaction Code] Mode Edit ข้อมูลที่เคยบันทึกไว้ เช่น WHT Rate กับ Show Trans code ไม่ตรงกับที่บันทึก}} \\
        ปัญหานี้เกิดจาก type ของ ฟิลด์ WHT Rate ไม่ตรงกับ Database \\
        เราเลยแก้ไขโดยการปรับ type ของฟิลด์ WHT Rate ให้ตรงกับ Database ก็จะสามารถดึงข้อมูลมาแสดงในโหมด Edit ได้ถูกต้อง
    \item \textbf{{GISLOCK-7694 [Other Charge] แก้ไขการค้นหาช่วงวันที่ Date From และ Date To ให้ตรงกับช่วงที่ค้นหา}} \\
        จากเดิมตอนเราเลือกใน format เป็น ISO 8601 อาจมีปัจจัยเรื่อง Timezone เข้ามาเกี่ยวข้องทำให้การค้นหาไม่ตรงกับช่วงวันที่ที่เลือก \\
        เราเลยแก้ไขโดยปรับเป็นรูปแบบ dd/mm/yyyy แทน ทีนี้ก็จะไม่มีปัญหาเรื่อง Timezone แล้ว ก็จะทำให้การค้นหาตรงกับช่วงวันที่ที่เลือก
    \item \textbf{{GISLOCK-7737 [Other Charge] Set up Transaction Code List กดปุ่ม Add แล้วไม่แสดงหน้าจอ Set up Transaction Code}} \\
        เรามีการปรับจากเดิมที่มีการใช้ getParam มาเป็นการใช้ getState \\
        พอเรากด Add มาจากหน้า List อาจทำให้ State ที่ส่งมาไม่ครบ เราเลยแก้ไขให้ตอนกด Add มันจะไม่เอา State อะไรมาส่งเลย ทำให้มัน Redirect มาที่หน้า Set up Transaction Code List ปกติ \\
        แก้ไขโดยการ set ค่า default ให้กับ State ที่ขาดหายไป 
    \item \textbf{{GISLOCK-7793 [Other Charge] ส่วน Share Department ขาดการแสดงคอลัมน์ Department Name ที่ตารางแสดงผล}} \\
        เพิ่มส่วนของ Department Name ในตารางแสดงผลของ Share Department \\
        โดยการดึงข้อมูลชื่อแผนกจากฐานข้อมูลมาแสดงในตาราง
    \item \textbf{{GISLOCK-7807 [Other Charge] OTC List กด link condition search และกดปุ่ม search ระบบแจ้งเตือน Invalid column name 'OTC\_BILLTO\_CD'.}} \\
        เกิดจากการที่เราปรับ Database ใหม่ ทำให้ชื่อคอลัมน์เปลี่ยนไป \\
        เราเลยแก้ไข SQL query ให้ตรงกับชื่อคอลัมน์ใหม่ที่ถูกต้อง
    \item \textbf{{GISLOCK-7847 [Other Charge] ฟิลด์ Vat Amount ส่วนของ Installment ไม่ Disable Field กรณี Transaction Code เป็น Non Vat ทั้งหมด (Edit จากหน้าจอ List เข้ามา)}} \\
        เพิ่มการเช็คข้อมูลตอนกด edit ว่าในตาราง Transaction มีแต่ Non Vat หรือไม่ \\
        ถ้ามีแต่ Non Vat ก็จะทำการ Disable Field Vat Amount ในตาราง Installment ให้อัตโนมัติ
    \item \textbf{{GISLOCK-7860 [Other Charge List] Action Recall ไม่สามารถใช้งานได้}} \\
        จากเดิมเราจะเป็นการเขียน mock up ขึ้นมา ทำให้ Action Recall ไม่สามารถใช้งานได้ \\
        เราเลยปรับให้ Action Recall สามารถใช้งานได้จริง โดยการเชื่อมต่อกับฐานข้อมูลและดึงข้อมูลที่ถูกต้องมาแสดง
\end{itemize}

\subsubsection{CR}
\begin{itemize}
    \setlength\itemsep{1em}
    \sloppy
    \item \textbf{{GISLOCK-6697 [OTC][Request OTC] ปรับการแสดงส่วน Invoice 1.Invoice Name 2.Invoice Address 3.Invoice Date}} \\
          ปรับการแสดง Invoice จากเดิมที่แสดงข้อมูลในรูปแบบ modal\\
            เป็นการแสดงข้อมูลในรูปแบบของ field ที่แสดงบนหน้าจอปกติแทน
    \item \textbf{{GISLOCK-6715[OTC][OTC List] แก้ไขชื่อคอลัมน์ และ ดึงข้อมูลมาแสดง หลังมีการ Demo 1 - 22/7/68}} \\
          ปรับจากกเดิมหลังจากที่ได้มีการ Demo กับทางผู้ใช้งานไปเมื่อวันที่ 22 กรกฎาคม 2568 \\
            โดยมีการเปลี่ยนแปลงดังนี้ \\
            1. เปลี่ยนชื่อคอลัมน์จาก Invoice Date เป็น Payment Due Date
    \item \textbf{{GISLOCK-6696 [OTC][OTE][Request OTC] ปรับการแสดง HQ/Branch และ Branch No.}} \\
          1.ปรับ HQ/Branch มาเป็นแสดงเป็นชื่อแทนโค้ด \\
          2.Branch No. เป็นค่าว่างเมื่อ HQ/Branch เป็นสำนักงานใหญ่ \\
          3.การแสดง Headquarter หรือ สำนักงานใหญ่อ้างอิงตาม Main Languaege
    \item \textbf{{GISLOCK-6698 [FA - Other Charge/Other Expense] หน้าจอ Request Other Charge ปรับ Contact Information เพิ่มคอลัมน์ Purpose For}} \\
          เพิ่มคอลัมน์ Purpose For ในส่วนของ Contact Information \\
            เพื่อให้ผู้ใช้สามารถระบุวัตถุประสงค์ในการติดต่อได้อย่างชัดเจนมากขึ้น 
    \item \textbf{{GISLOCK-6699 [OTC][OTE][Request OTC] ส่วน Other Charge Transaction}} \\
          1.Currency กรณีเป็น Thai หรือ Local default ค่าเป็ฯ THB \\
          2.Exchange Rate ดึงค่ามาจากค่ากลาง  \\
          3.ปรับ UI ของ Description ของ Transaction \\
          4.ที่โปรแกรม Request Information :\\
            - สลับฟิลด์ Insured มาแสดงก่อน Type of Insurance \\
            - Type of Insurance ปรับมาเป็น Dropdown List \\
            - Brokerage Amount คำนวณ Auto เมื่อเรา ระบุ Brokerage \%
    \item \textbf{{GISLOCK-7196 [All Screen] แก้ไข Icon ในปุ่ม Save Draft ให้เหมือนกัน}} \\
          ปรับไอคอนในปุ่ม Save Draft ให้มีความสอดคล้องและเหมือนกันในทุกหน้าจอ \\
            เพื่อให้ผู้ใช้สามารถจดจำและใช้งานได้ง่ายขึ้น
    \item \textbf{{GISLOCK-7390 [Date] เพิ่มเรื่องการ validate วันที่ From ToGISLOCK-7398 [Other Charge] เพิ่มเรื่องการ validate วันที่ From To}} \\
          ปรับไปดึงฟังก์ชันการตรวจสอบวันที่ From และ To จากฟังก์ชั่นกลาง \\ 
            เพื่อให้การตรวจสอบมีความถูกต้องและสอดคล้องกับมาตรฐานที่กำหนดไว้
    \item \textbf{{GISLOCK-7617 [Other Charge] ปรับชื่อฟิลด์ ในหน้าจอ และ DB}} \\
            ปรับชื่อฟิลด์ในหน้าจอและฐานข้อมูลให้มีความสอดคล้องและเข้าใจง่ายขึ้น \\
                เพื่อให้ผู้ใช้สามารถใช้งานและเข้าใจข้อมูลได้อย่างถูกต้อง หลังจากที่ได้รับข้อเสนอแนะจากผู้ใช้งาน
\end{itemize}