\subsection{Angular}
Angular คือเฟรมเวิร์กสำหรับฝั่ง Front-end ที่ใช้ในการพัฒนา Web Application แบบ Single Page Application (SPA) 
ซึ่งช่วยให้นักพัฒนาสามารถสร้างส่วนติดต่อผู้ใช้ (User Interface) ที่มีประสิทธิภาพ ตอบสนองเร็ว 
และสามารถดูแลรักษา (Maintainable) ได้ง่าย โดย Angular ออกแบบมาให้ทำงานแบบ Component-Based 
และมีระบบจัดการ Data Binding และ Dependency Injection ที่แข็งแกร่ง\

\textbf{คุณสมบัติเด่นของ Angular}
\begin{itemize}
    \item Component-Based Architecture: แบ่งส่วนของหน้าเว็บออกเป็น Component ย่อย ๆ ทำให้โค้ดเป็นระเบียบและนำกลับมาใช้ซ้ำได้
    \item Two-Way Data Binding: ช่วยให้ข้อมูลระหว่าง Model และ View เชื่อมโยงกันแบบอัตโนมัติ ลดความซับซ้อนในการอัปเดต UI
    \item Dependency Injection: จัดการ Service และ Object ที่จำเป็นได้อย่างมีประสิทธิภาพ เพิ่มความยืดหยุ่นในการพัฒนา
    \item Angular CLI: เครื่องมือ Command Line ที่ช่วยสร้าง Project, Component, Service และ Build แอปพลิเคชันได้รวดเร็วและเป็นมาตรฐาน
\end{itemize}

\textbf{บริบทของ Angular ใน Fullstack Development}
\begin{itemize}
    \item ทำหน้าที่เป็น Front-end ที่รับผิดชอบการแสดงผลและการโต้ตอบกับผู้ใช้
    \item เรียกใช้งาน RESTful APIs จากฝั่ง Backend (เช่น Spring Boot) เพื่อรับและส่งข้อมูลในรูปแบบ JSON
    \item จัดการ State ของแอปพลิเคชัน และอัปเดตข้อมูลบนหน้าเว็บโดยไม่ต้อง Reload ทั้งหน้า (SPA)
    \item ใช้ร่วมกับระบบ Routing ภายในเพื่อสลับหน้าแบบ Dynamic โดยไม่ต้องโหลดจากเซิร์ฟเวอร์ใหม่
\end{itemize}