% % Example
% \newcommand{\otherchargeCards}{
%     XDO-1001 & Fix calculation bug \\
%     XDO-1002 & Update UI label \\
%     XDO-1003 & Add new feature for export \\
%     XDO-1004 & Change request for report
% }

% \newcommand{\jiralink}[1]{\texttt{#1}}

\setcounter{secnumdepth}{3}

\chapter{\ifenglish Cooperative Details\else รายละเอียดเกี่ยวกับการทำงาน / ผลการปฏิบัติงาน \fi}

\section{\ifenglish Foundation of Fullstack\else ปรับความรู้พื้นฐานของการเป็น Fullstack\fi}
แนวทางในการเริ่มต้นทำงานในสายงาน Fullstack จำเป็นต้องมีการศึกษาและปรับพื้นฐานความรู้ที่สำคัญเพื่อให้แน่ใจว่าพร้อมสำหรับการทำงานจริง  
เนื่องจาก Fullstack เป็นสายงานที่มีความใหม่และมีการพัฒนาอย่างต่อเนื่องในวงการซอฟต์แวร์ โดยหัวข้อที่ได้รับมอบหมายให้ศึกษาเพื่อเตรียมความพร้อมจะประกอบด้วย

\begin{itemize}
    \item \textbf{Angular}: Framework สำหรับพัฒนา Front-end Web Application แบบ Single Page Application (SPA) ช่วยให้การจัดการ Component, Template, และ Data Binding มีประสิทธิภาพ
    \item \textbf{Java Spring Boot}: Framework สำหรับพัฒนา Back-end Application ช่วยให้สร้าง REST API และจัดการ Business Logic ได้อย่างเป็นระบบ
    \item \textbf{SQL Server}: ระบบฐานข้อมูลเชิงสัมพันธ์ (Relational Database) สำหรับจัดเก็บและจัดการข้อมูลของแอปพลิเคชัน
\end{itemize}

ทั้งนี้การศึกษาหัวข้อเหล่านี้มีระยะเวลาประมาณ 2-4 สัปดาห์ และในท้ายที่สุดจะต้องมีการนำเสนอสิ่งที่ได้เรียนรู้
ให้กับพี่ ๆ ในทีมได้ฟังและประเมิณว่าพร้อมที่จะทำงานจริงหรือไม่ อย่างไรก็ตามรายละเอียดในหัวข้อย่อยต่าง ๆ หลังจากนี้จะเป็นการนำเสนอสิ่งที่ได้เรียนรู้และได้นำมาประยุกต์ใช้ในการทำงานจริง
ส่วนหัวข้อนอกเหนือจากที่จะกล่าวถึงก็สำคัญไม่น้อยเช่นกันแต่จะข้อนำเสนอ Documentation ที่ได้ทำสรุปการเรียนรู้มาแล้วนั้นในส่วนภาคผนวก

\clearpage
\import{chapters/detail_sections}{angular.tex}
\clearpage
\import{chapters/detail_sections}{springboot.tex}
\clearpage
\import{chapters/detail_sections}{sql_server.tex}
\clearpage

\section{\ifenglish TOR\else TOR\fi}

เนื่องจากการมาสหกิจศึกษาจำเป็นต้องมีการประเมินที่เข้มงวด ดังนั้นจึงได้จัดทำ TOR ขึ้นเพื่อเป็นมาตรฐานและข้อตกลงในการทำงาน  
และประเมินผลร่วมกับบริษัทและอาจารย์ เพื่อให้มั่นใจว่างานสามารถทำได้ตามเป้าหมายของ Fullstack Project  

\begin{itemize}
    \item ทำงานในลักษณะของ Task-based โดยกำหนดจำนวนขั้นต่ำของ Task ไว้ที่ 60 tasks  
    เนื่องจากทีม Fullstack ทำงานตามแนวทาง Agile โดยแบ่งงานเป็น Sprint  
    ทุก Sprint จะมีการ Planning เพื่อกำหนดลำดับความสำคัญของ Task  
    และมีการประชุม Weekly-Update ทุกวันพุธเพื่อติดตามความคืบหน้า ทำให้การทำงานมีความรอบคอบและสามารถปรับเปลี่ยนได้ตามความต้องการของ Product Owner และลูกค้า
\end{itemize}
ดังนั้นหัวข้อถัดไปจะเป็นการนำเสนอ Task งานที่ได้รับมอบหมายทั้งหมด เพื่อแสดงให้เห็นว่าสามารถบรรลุเป้าหมายของ TOR ได้ตามแผนงาน Agile

\section{\ifenglish Responsibility Task\else งานที่ได้รับมอบหมาย\fi}

สืบเนื่องจากงานที่ได้รับมอบหมายจะเป็นในลักษณะ Task งาน  
ดังนั้นใน Section นี้จะนำเสนอ Task งานทั้งหมด พร้อมรายละเอียดของแต่ละ Task งาน  
โดยจะแบ่งออกเป็น 3 กลุ่มหลัก ดังนี้:

\begin{itemize}
    \item \textbf{TASK}: งานพัฒนาฟีเจอร์ใหม่หรือทำงานตาม requirement ที่กำหนด
    \item \textbf{CR (Change Request)}: งานปรับปรุงหรือปรับแก้ตามคำขอเพิ่มเติมจากผู้ใช้งานหรือ Product Owner
    \item \textbf{DEFECT}: งานแก้ไขบัคหรือปัญหาที่เกิดขึ้นในระบบ
\end{itemize}
การแบ่งกลุ่ม Task ดังกล่าวช่วยให้สามารถติดตามความคืบหน้าในลักษณะ Agile ได้อย่างชัดเจน  
และสะท้อนถึงความสามารถในการจัดการงานทั้งการพัฒนา การปรับปรุง และการแก้ไขปัญหา

% เรียก Module ทั้งหมด
\subsection{Other Charge Module}

\subsubsection{Task}
\begin{itemize}
    \setlength\itemsep{1em}
    \item \textbf{\jiralink{GISLOCK-5943 [FA-Other charge request] จัดทำหน้าจอ Other charge request}} \\
          เป็นการเพิ่มโปรแกรมใหม่ในแผนกของการเงิน โดยการเป็นสรา้งโปรแกรมที่ชื่อว่า Other Charge Request และ Other Charge Request List\\
          Other Change Request : เป็นโปรแกรมเกี่ยวกับการเรียกค่าใช้จ่ายเพิ่มเติมนอกเหนือจากเบี้ยประกัน \\
          Other Change Request List : เป็นหน้าที่เอาไว้รายการที่เราสร้างไว้
    \item \textbf{\jiralink{GISLOCK-6518 [FA- Other charge request] ปรับ feature ทำหน้าจอ Other charge request}} \\
          description
    \item \textbf{\jiralink{GISLOCK-6694 [OTC][Request OTC] ปรับการแสดงส่วน Invoice 1.Invoice Name 2.Invoice Address 3.Invoice Date}} \\
          description
    \item \textbf{\jiralink{GISLOCK-6755 [Set up Transaction Code] เพิ่มหน้าจอ Set up Transaction Code}} \\
          description
    \item \textbf{\jiralink{GISLOCK-7019 [Other Charge] เพิ่มปุ่ม Update (สำหรับ Refresh) ข้อมูล Installment และ ShareDept}} \\
          description
    \item \textbf{\jiralink{GISLOCK-7106 [Other Charge] แก้ไขโปรแกรม Other charge Request List เพิ่มเงื่อนไขในการ query เพื่อค้นหาข้อมูลโดยกรองเฉพาะ Entry\_by ที่ตรงกับ User Login}} \\
          description
    \item \textbf{\jiralink{GISLOCK-7200 [Other Charge] แก้ไขโปรแกรม Charge Request ส่วนของการ Share dept \, Installment}} \\
          description
    \item \textbf{\jiralink{GISLOCK-7289 [Other Charge] Set up Transaction Code เพิ่ม DropdownFlag}} \\
          description
    \item \textbf{\jiralink{GISLOCK-7617 [Other Charge] ปรับชื่อฟิลด์ ในหน้าจอ และ DB}} \\
          description
    \item \textbf{\jiralink{GISLOCK-7688 [Other Charge] Other Charge Request Booking}} \\
          description
    \item \textbf{\jiralink{GISLOCK-7689 [Other Charge] Other Charge Request บันทึกขาคู่}} \\
          description
\end{itemize}

\subsubsection{Defect}
\begin{itemize}
    \setlength\itemsep{1em}
    \item \textbf{\jiralink{GISLOCK-5457 [Other Charge] ปรับการแสดงผลข้อมูลในส่วน Bill to ให้ไปในทางเดียวกับ AP}} \\
          description
    \item \textbf{\jiralink{GISLOCK-6189 [Other Chare/Expense]Standard Search Search like เวลาเก็บ Data ใน DB ให้คั่นด้วย Space Bar}} \\
          description
    \item \textbf{\jiralink{GISLOCK-6561 [Other Charge] Pop up \: Bill To แสดงข้อมูลเดียวกัน ซ้ำๆ}} \\
          description
    \item \textbf{\jiralink{GISLOCK-6563 [Other Charge] Contact Information จะต้องดึงข้อมูลมาจาก Contact Person ของ Client หรือ Partner มาแสดงโดย link up กับ Bill to ที่ได้เลือก}} \\
          description
    \item \textbf{\jiralink{GISLOCK-6566 [Other Charge] ปรับ wording ของชื่อฟิลด์ ทั้งหน้า Request และ List}} \\
          description
    \item \textbf{\jiralink{GISLOCK-6568 [Other Charge] ตรวจสอบการ Save Draft ใบงานและการแสดงค่าเริ่มต้นฟิลด์ Status}} \\
          description
    \item \textbf{\jiralink{GISLOCK-6571 [Other Charge] ส่วน Share Department ปรับการแสดงปุ่ม Add \, Clear และเมื่อเลือกข้อมูล Department รูปแบบฟอร์มต้องไม่ขยับไปมา}} \\
          description
    \item \textbf{\jiralink{GISLOCK-6575 [Other Charge] การใช้งานปุ่ม View ในส่วน Other Charge Transaction}} \\
          description
    \item \textbf{\jiralink{GISLOCK-6576 [Other Charge] กรณี Total Amount ใน Transaction และ Installment มียอดไม่เท่ากัน เมื่อ Save \, Submit จะต้องมี Pop up ถาม}} \\
          description
    \item \textbf{\jiralink{GISLOCK-6577 [Other Charge List] ปรับการใช้งาน Request Date From / To}} \\
          description
    \item \textbf{\jiralink{GISLOCK-6578 [Other Charge List] ฟิลด์ Bill To กดแล้วไม่แสดง Pop up และขอความหมายของ Request No. Running No. C2568XXXXX}} \\
          description
    \item \textbf{\jiralink{GISLOCK-6583 [Other Charge List] ไม่แสดงข้อมูล Bill to Name}} \\
          description
    \item \textbf{\jiralink{GISLOCK-6589 [Other Charge List] การใช้งานและความหมายของ Submit Date}} \\
          description
    \item \textbf{\jiralink{GISLOCK-6591 [Other Charge] ปรับชื่อฟิลด์ Entry By และ Entry Date ให้กับหน้าจอ Request และ List}} \\
          description
    \item \textbf{\jiralink{GISLOCK-6592 [Other Charge List] ไม่แสดงข้อมูลในคอลัมน์ Invoice Date}} \\
      description
    \item \textbf{\jiralink{GISLOCK-6649 [Other Charge][Request Other Charge] ปรับการทำงานส่วน Contact Information ให้มีฟังก์ชันการทำงานเหมือนกับ AP}} \\
        description
    \item \textbf{\jiralink{GISLOCK-6675 [Other Charge][Request Other Charge] ส่วน Request Information ทำการ Edit ข้อมูลในตารางแต่ระบบไม่ Update ตาม}} \\
        description
    \item \textbf{\jiralink{GISLOCK-6773 [Other Charge][Request Other Charge] Submit ใบงาน ระบบแสดง The sum of all tables is not equal.}} \\
        description
    \item \textbf{\jiralink{GISLOCK-6877 [OTC][Request Other Charge] กรณีเลือก Edit รายการเพื่อตรวจสอบข้อมูล แต่ Entry Date แสดงเป็นวันที่ปัจจุบัน ซึ่งไม่ถูกต้องให้แก้ไขเป็นวันที่บันทึกใบงานตั้งต้น}} \\
        description
    \item \textbf{\jiralink{GISLOCK-6878 [OTC][Request Other Charge] กรณีเลือก Invoice Name ฟิลด์ Invoice Address จะต้องแสดงข้อมูลที่สัมพันธ์กับ Name ที่เลือก}} \\
        description
    \item \textbf{\jiralink{GISLOCK-6945 [Other Charge List] ปรับการแสดง Location และ Nationality กรณี Bill To เป็นข้อมูลเก่า โดยระบบต้องแสดงข้อมูลที่ได้เลือกจากตอนบันทึก}} \\
        description
    \item \textbf{\jiralink{GISLOCK-6953 [Other Charge] 1.แก้ไขการแสดง Request Information 2.แก้ไขการแสดง Currency/Exchange Rate 3.แก้ไขการแสดง Bill to Code}} \\
        description
    \item \textbf{\jiralink{GISLOCK-7031 [Other Charge] ปรับการทำงานปุ่ม Submit และ Save Draft}} \\
        description
    \item \textbf{\jiralink{GISLOCK-7064 [Other Charge] เมื่อ Submit and Save Draft ข้อมูลที่บันทึกไว้ในหน้าจอต้องไม่เคลียร์และ Redirect to List}} \\
        description
    \item \textbf{\jiralink{GISLOCK-7144 [Other Charge List] ยอด Amount แสดงไม่ตรงกับการบันทึกในหน้า Request}} \\
        description
    \item \textbf{\jiralink{GISLOCK-7215 [Other Charge] ตรวจสอบการคำนวณ Payment Due Date ให้ตรงตามเงื่อนไข}} \\
        description
    \item \textbf{\jiralink{GISLOCK-7219 [Other Charge] เพิ่มการแจ้งเตือน Share Department และ Installment กรณีที่ Share (\%) มีจำนวน 0.00 และมากกว่า 100.00}} \\
        description
    \item \textbf{\jiralink{GISLOCK-7230 [Other Charge] แก้ไขการกดปุ่ม Clear และ เพิ่ม Popup has been saved ให้กับการเพิ่มข้อมูลลงตาราง}} \\
        description
    \item \textbf{\jiralink{GISLOCK-7241 [Other Charge] ลบรายการ Transaction ออกไปแล้ว ข้อมูลส่วน Installment และ ปุ่ม Update ยังแสดงอยู่ในหน้าจอ}} \\
        description
    \item \textbf{\jiralink{GISLOCK-7243 [Other Charge] กดปุ่ม Update แล้วไม่ Update ข้อมูล กรณี Installment และ Share Dept มีข้อมูลทั้งคู่}} \\
        description
    \item \textbf{\jiralink{GISLOCK-7258 [Other Charge] เพิ่มการปรับ \% ของ Share Department และ Installment}} \\
        description
    \item \textbf{\jiralink{GISLOCK-7395 [Other Charge] เมื่อเลือก Vat Calculation = Non Vat ให้ Disable Field 1.Vat Type 2.Vat Rate \% 3.Vat Amount}} \\
        description
    \item \textbf{\jiralink{GISLOCK-7419 [Other Charge] ส่วน Contact Information เพิ่มเงื่อนไขเมื่อไม่มีการเลือก Code ให้ Disable ส่วน Contact ไว้ก่อนเสมอ}} \\
        description
    \item \textbf{\jiralink{GISLOCK-7439 [Other Charge] Vat Rate \% = 0 ให้ Disable Vat Amount}} \\
        description
    \item \textbf{\jiralink{GISLOCK-7487 [Other Charge] ตรวจสอบการค้นหาข้อมูลแบบ ค้นหามีช่องว่างระหว่างคำ}} \\
        description
    \item \textbf{\jiralink{GISLOCK-7494 [Other Charge] ปรับตำแหน่งการแสดงปุ่ม Search และ Clear}} \\
        description
    \item \textbf{\jiralink{GISLOCK-7498 [Other Charge] Set up Transaction Code แก้ไขตามรายการ}} \\
        description
    \item \textbf{\jiralink{GISLOCK-7529 [Other Charge] โปรแกรม Set up Transaction Code ทำการ Delete รายการในตารางแล้ว รายการไม่ถูกลบไป}} \\
        description
    \item \textbf{\jiralink{GISLOCK-7554 [Other Charge] ลบรายการที่มี Vat Amount ที่ส่วน Transaction ออกไปแล้ว แต่ในส่วน Installment ยังค้างยอด Vat Amount เก่าไว้}} \\
        description
    \item \textbf{\jiralink{GISLOCK-7587 [Other Charge] Set up Transaction Code เปลี่ยนจากปุ่ม Submit/Clear เป็น Save/Clear}} \\
        description
    \item \textbf{\jiralink{GISLOCK-7626 [Other Charge] เปลี่ยนจากการส่งข้อมูลแบบ queryParams เป็นแบบ getState}} \\
        description
    \item \textbf{\jiralink{GISLOCK-7628 [Other Charge] ปรับ status เป็นตัวกลางเพื่อให้ sort จะได้ เนื่องจากตอนนี้เป็นการใช้เงื่อนไขในการแปลงจาก code เป็น word และนำมาแสดง ไม่ใช่การ Join}} \\
        description
    \item \textbf{\jiralink{GISLOCK-7629 [Other Charge] ลบข้อมูลในตาราง Installment แล้วในตาราง Transaction มีแต่เคส VATNO ระบบไม่ Disable Field Vat ในส่วน Installment ให้}} \\
        description
    \item \textbf{\jiralink{GISLOCK-7631 [Other Charge][Set up Transaction Code] Mode Edit ข้อมูลที่เคยบันทึกไว้ เช่น WHT Rate กับ Show Trans code ไม่ตรงกับที่บันทึก}} \\
        description
    \item \textbf{\jiralink{GISLOCK-7694 [Other Charge] แก้ไขการค้นหาช่วงวันที่ Date From และ Date To ให้ตรงกับช่วงที่ค้นหา}} \\
        description
    \item \textbf{\jiralink{GISLOCK-7737 [Other Charge] Set up Transaction Code List กดปุ่ม Add แล้วไม่แสดงหน้าจอ Set up Transaction Code}} \\
        description
    \item \textbf{\jiralink{GISLOCK-7793 [Other Charge] ส่วน Share Department ขาดการแสดงคอลัมน์ Department Name ที่ตารางแสดงผล}} \\
        description
    \item \textbf{\jiralink{GISLOCK-7807 [Other Charge] OTC List กด link condition search และกดปุ่ม search ระบบแจ้งเตือน Invalid column name 'OTC\_BILLTO\_CD'.}} \\
        description
    \item \textbf{\jiralink{GISLOCK-7847 [Other Charge] ฟิลด์ Vat Amount ส่วนของ Installment ไม่ Disable Field กรณี Transaction Code เป็น Non Vat ทั้งหมด (Edit จากหน้าจอ List เข้ามา)}} \\
        description
    \item \textbf{\jiralink{GISLOCK-7860 [Other Charge List] Action Recall ไม่สามารถใช้งานได้}} \\
        description
    \item \textbf{\jiralink{GISLOCK-7932 [Other Charge] Billing Information กรณี Charge to Code เป็น All Purpose ระบบไม่แสดงข้อมูลให้อัตโนมัติ}} \\
        description
\end{itemize}

\subsubsection{CR}
\begin{itemize}
    \setlength\itemsep{1em}
    \item \textbf{\jiralink{GISLOCK-6694 [OTC][Booking] แก้ไขโปรแกรมหลัง Demo1 with Champion GISLOCK-6715[OTC][OTC List] แก้ไขชื่อคอลัมน์ และ ดึงข้อมูลมาแสดง หลังมีการ Demo 1 - 22/7/68}} \\
          description
    \item \textbf{\jiralink{GISLOCK-6696 [OTC][OTE][Request OTC] ปรับการแสดง HQ/Branch และ Branch No.}} \\
          description
    \item \textbf{\jiralink{GISLOCK-6698 [FA - Other Charge/Other Expense] หน้าจอ Request Other Charge ปรับ Contact Information เพิ่มคอลัมน์ Purpose For}} \\
          description
    \item \textbf{\jiralink{GISLOCK-6699 [OTC][OTE][Request OTC] ส่วน Other Charge Transaction}} \\
          description
    \item \textbf{\jiralink{GISLOCK-7196 [All Screen] แก้ไข Icon ในปุ่ม Save Draft ให้เหมือนกัน}} \\
          description
    \item \textbf{\jiralink{GISLOCK-7390 [Date] เพิ่มเรื่องการ validate วันที่ From ToGISLOCK-7398 [Other Charge] เพิ่มเรื่องการ validate วันที่ From To}} \\
          description
\end{itemize}
\subsection{Client/Partner/Setting/General Module}

\subsubsection{Defect}
\begin{itemize}
    \setlength\itemsep{1em}
    \item \textbf{{GISLOCK-4938 [Transfer Prospect to Client][Client Amendment] กด Edit > Update ระบบ Clear ข้อมูล Attachments}} \\
          เกิดจากการที่ตอนแรกมันเซ็ต setAtm\_puid\_ref\_file เป็น null เลยส่งค่า invalidData ไปเลยทำให้แสดงค่านั้น แก้ไขโดยการเซ็ตค่าก่อนจะส่งไปทำให้หาเจอ แล้วก็สามารถแสดงได้ปกติ      
    \item \textbf{{GISLOCK-5128 [New Prospect/Client] กรณี Duplicate งานต่าง Department แล้วมีการแนบ Attachment ระบบไม่สามารถ Preview File ได้}} \\
          เกิดจากการที่ตอนแรกไม่มีการ set department เลยทำให้การส่ง atm\_can\_preview ไปเป็น N ซึ่งเปิดดูไม่ได้ แต่พอเรา setDept ใหม่ คือเราที่เข้ามา เลย พอเรา setDept เป็นคนที่เราแนบเข้ามา มันเลยไปเช็ค dept ใน database มันเลยตรงกันทำให้ ส่ง atm\_can\_preview มาให้หน้าบ้านเลยสามารถ preview ได้
    \item \textbf{{GISLOCK-5157 [Partner Amendment] เปลี่ยนชื่อ Field จาก Thai Name, English Name เป็น Partner Name}} \\
          เเปลี่ยนชื่อ Field จาก Thai Name, English Name เป็น Partner Name (TH) , Partner Name (EN)
    \item \textbf{{GISLOCK-5170 [Partner Amendment][Partner Previous Details] แก้ไขชื่อ Attachments และ Address Location}} \\
          เปลี่ยนชื่อ Field จาก Address , Attachment เป็น Address Location , Attachments
    \item \textbf{{GISLOCK-5230 [Partner Amendment] ระบบไม่ Disable Field Payment Method Detail}} \\
          เพิ่มการ Disable Field Payment Method Detail
    \item \textbf{{GISLOCK-5231 [Partner] ระบบไม่แสดงข้อมูล Withholding Tax Rate}} \\
          งานนนี้เป็นงานแก้ไขโดยการเปลี่ยนไปใช้ฟังก์ชันกลางที่ทำไว้แล้ว อันเดิมเป็นเรียกฟังก์ชันเก่าเลยไม่แสดงข้อมูล Withholding Tax Rate พอเราปรับไปใช้ฟังก์ชั่นใหม่ก็จะแสดงมาปกติ     
    \item \textbf{{GISLOCK-5265 [Approve Client][Approve Partner] ระบบแสดง Attached By ไม่ถูกต้อง (ติดการ์ด 5621)}} \\
          งานนนี้เป็นงานแก้ไขเงื่อนไขการเช็ค typeof partnerTypeValueList === 'object' เพิ่มไปด้วย ที่เหลือจะเป็นงานที่ไปปรับตรงของ GISLOCK-5261 
    \item \textbf{{GISLOCK-5302 [Partner Amendment] แก้ไขการแสดงข้อมูล Contact Person ของปุ่ม Partner Previous Detail}} \\
          เปลี่ยนชื่อ Field จาก Relation to Client เป็น Relation to Partner
    \item \textbf{{GISLOCK-5699 [Setting][Program on Module] Edit รายการเเล้ว Error}} \\
          มี 2 กรณี คือ \\
          1. Save แล้ว Error เพราะมันเอาชื่อที่เราแก้ไปเซ็ตให้ตรง menu แล้วมันเอาชื่อนั้นมาเช็คแล้วเจอข้อมูลเลยส่งไป แก้ไขโดยการเปลี่ยนเงื่อนไข จากเดิมใช้ชื่อ ตอนนี้ปรับมาใช้ prog\_id และ folder\_id มาเช็คแทน เพราะมันคนละส่วนกัน \\ 
          2. เพิ่มกรณีเมื่อเรากด Edit เข้ามาแล้วกดปุ่ม Clear ก็จะนำข้อมูลที่แก้ไขไปก่อนหน้านี้กลับมาได้
    \item \textbf{{GISLOCK-5752 [Partner][Partner Amendment] Accordion General Info ระบุข้อมูลที่ Company Name (EN) เป็น Lockton ระบบไม่ Auto เลือกข้อมูล Inter Company (LWT Group) เป็น Yes}} \\
          เพิ่มเงื่อนไขในการเช็ค Company Name (EN) ว่าเป็น Lockton หรือเปล่า ถ้าใช่ก็จะ set ค่า Inter Company (LWT Group) เป็น Yes ให้อัตโนมัติในหน้า Partner Amendment ด้วย
    \item \textbf{{GISLOCK-5753 [Partner Amendment] ไม่สามารถกด Sort Description ที่ Popup Copy Address From Address Location ได้}} \\
          เกิดจากการที่เราส่งตัว formcontrol ไปผิด เดิมส่ง descriptionEn มาเป็น description แทน
    \item \textbf{{GISLOCK-5761 [General Setup][Document Running] Icon Warning ต้องเป็นเครื่องหมายตกใจ สีเหลือง}} \\
          งานนนี้เป็นงานแก้ไขโดยการเปลี่ยนไอค่อนของ modal แจ้งเตือนใหม่
    \item \textbf{{GISLOCK-7013 [Parent/Group] Add Parent แล้วไม่ติ้ก Auto Create Client Group ระบบแสดงไม่ถูกต้อง}} \\
          งานนนี้เป็นงานแก้ไขหน้า UI ตรงการแสดงผล ตอนแรกมันจะแสดงภายในกรอบทำให้แสดงไม่ครบ เราปรับมาใช้ appendTo: 'body' เพื่อให้แสดงข้อมูลได้ถูกต้อง
    \item \textbf{{GISLOCK-7052 [New Prospect/Client] ระบบไม่แสดง Effective Date ที่ Introducer Info}} \\
          เกิดจากการที่ type ของ Effective Date ตรงหน้าบ้านประกาศไม่ตรงกับ type ที่หลังบ้านส่งมาให้
    \item \textbf{{GISLOCK-7056 [Document Set] เมื่อกด Add Document Set ระบบไม่สามารถกดปุ่ม Save ได้}} \\
          เกิดจากการที่เราทำ action อื่นก่อนมา add เช่น update มันจำข้อมูลเดิม ทำให้ Save ไม่ได้ แก้ไขโดยการ reset ข้อมูลก่อนหน้า ก่อนจะ add เพื่อให้ข้อมูลเป็นอันปัจจุบัน
    \item \textbf{{GISLOCK-7096 [Prospect/Client][Contact Person] กดเลือก All Purpose ให้ระบบปิด Dropdown}} \\
          เมื่อเรากดเลือก All Purpose ให้ระบบปิด Dropdown
    \item \textbf{{GISLOCK-7154 [Country] ระบบไม่สามารถ Search ได้ และไม่สามารถ Delete ได้}} \\
          มี 2 กรณี คือ \\
          1. ไม่สามารถ Search ได้ สาเหตุมาจากที่หลังบ้านเราใช้ SQL ที่ผิด เดิมใช้ LIKE ที่ไม่มี \%\% ครอบ ทำให้ค้นหาไม่เจอ เราเลยเพิ่ม \%\% ครอบเข้าไป \\ 
          2. ไม่สามารถ Delete ได้ เพราะตอน get ข้อมูลมาไม่ได้เลขไอดีที่ตรงกับตัวที่จะลบ พอเรากด delete หน้าบ้านมันเลยไม่รู้ว่าต้องลบตัวไหน แก้ไขโดยการ get ไอดีมาด้วย แล้วสามารถ delete ได้ปกติ
\end{itemize}

\subsubsection{CR}
\begin{itemize}
    \setlength\itemsep{1em}
    \item \textbf{{GISLOCK-5176 [Partner Amendment] เพิ่ม Field Start Date}} \\
          งานนนี้เป็นงานปรับปรุงจากอันเดิมคือในส่วนของหน้า Partner Amendment เราจะเพิ่มฟิลด์ Start Date เข้าไป
    \item \textbf{{GISLOCK-5769 [General Setup][Standard Code] แก้ไขหน้าจอตาม New Requirement}} \\
          งานนนี้เป็นงานปรับปรุงจากอันเดิมคือในส่วนของหน้า Standard Code เราจะเพิ่มฟิลด์ Create By , Create Date , Update By , Update Date เข้าไป
    \item \textbf{{GISLOCK-5773 [General Setup][Document Running Format ] แก้ไขหน้าจอตาม New Requirement}} \\
          งานนนี้เป็นงานปรับปรุงจากอันเดิมคือในส่วนของหน้า Document Running Format เราจะเพิ่มฟิลด์ Create By , Create Date , Update By , Update Date เข้าไป
    \item \textbf{{GISLOCK-5774 [General Setup][Document Running Sequence] แก้ไขหน้าจอตาม New Requirement}} \\
          งานนนี้เป็นงานปรับปรุงจากอันเดิมคือในส่วนของหน้า Document Running Sequence เราจะเพิ่มฟิลด์ Create By , Create Date , Update By , Update Date เข้าไป
    \item \textbf{{GISLOCK-5832 [Client][Partner] ระบบไม่สามารถ Preview File กรณีเป็น View Mode ได้}} \\
          งานนนี้เป็นงานปรับปรุงโดยการเพิ่มการ preview file ใน view mode เขาไป
    \item \textbf{{GISLOCK-7065 [LWT Bank Account] แก้ไขโปรแกรมตาม New Requirement}} \\
          ปรับจากเดิมที่เป็น Modal มาเป็นหน้า page งานนี้เป็นการทำหน้า Ui ของ LWT Bank Account
    \item \textbf{{GISLOCK-7098 [New Prospect/Client][New Partner] กรณีเลือก All Purpose ให้ Disable ตัวเลือกอื่นๆ}} \\
          งานนนี้เป็นงานปรับปรุงจากอันเดิมคือในส่วนของ Contact จะมี Dropdown ที่เลือก Purpose กรณีที่เราเลือก "All Purpose" เราจะ Disable ฟิลด์อื่นแล้วก็ปิด Dropdown
    \item \textbf{{GISLOCK-7223 [Prospect/Client] เพิ่มข้อมูลสถานะเกี่่ยวกับการบันทึกข้อมูล Prospcet/Client}} \\
          เพิ่ม Field ของ Entry By , Entry Date , Modification By , Modification Date ที่หน้าของ New Client , Client Amendment , Transfer Prospect to Client
    \item \textbf{{GISLOCK-7224 [Partner] เพิ่มข้อมูลสถานะเกี่่ยวกับการบันทึกข้อมูล Partner \& Market Security}} \\
          เพิ่ม Field ของ Entry By , Entry Date , Modification By , Modification Date ที่หน้าของ New Partner , Partner Amendment , Market Security 
\end{itemize}


\subsection{FA-AR Module}

\subsubsection{Defect}
\begin{itemize}
    \setlength\itemsep{1em}
    \item \textbf{{GISLOCK-5808 [FA-AR][ใบนัดรับเงิน] Appointment Information กรณีเปลี่ยน Page ที่ไม่ใช่ 1 ระบบเรียงลำดับข้อมูลไม่ถูกต้อง}} \\
          เกิดจากการที่หลังบ้านไม่ได้เขียน SQL เพื่อเรียงข้อมูล เลยทำให้ตอนที่เรากดไปหน้าใหม่ข้อมูลมันเลยเรียงไม่ถูกต้อง
    \item \textbf{{GISLOCK-5432 [FA-AR][ใบนัดรับเงิน] Appointment No wording ผิด}} \\
          งานนี้เราแก้แค่ wording เป็น Appointment No.     
    \item \textbf{{GISLOCK-5424 [FA-AR][Appointment List] placeholder wording ผิด}} \\
          งานนี้เป็นการปรับ placeholder เป็นคำใหม่ 'Client Code/Client Group/Bill Statement No '
    \item \textbf{{GISLOCK-5622 [FA-AR][New ใบนัดรับเงิน] ชื่อคอลัมน์ PaymentDueDate ไม่เว้นวรรค}} \\
          งานนนี้เป็นงานแก้การเว้นวรรคำ จาก 'PaymentDue Date' เป็น 'Payment Due Date'
\end{itemize}


\section{\ifenglish Benefit\else สวัสดิการที่ได้รับ\fi}
บริษัท G-Able ให้ความสำคัญกับเรื่องของสวัสดิการที่ดีให้กับพนักงาน โดยเฉพาะนักศึกษาสหกิจศึกษาที่เข้ามาทำงานในบริษัท
ถึงแม้จะไม่ได้เป็นพนักงานประจำ แต่ก็ได้รับสวัสดิการที่ดีจากบริษัทอย่างเช่น
\begin{itemize}
      \item ทำงาน 5 วัน / สัปดาห์ ทำงานแบบ Work from home \\(09:00 - 18:00)
      \item โน๊ตบุ๊คสำหรับใช้ทำงานระหว่างฝึกงาน
      \item เบี้ยเลี้ยงในการทำงาน 500 บาท / วันทำงาน (ไม่นับวันลา หรือวันหยุด)
      \item วันหยุดประจำปีตามประเทศไทย และวันหยุดพิเศษตามประเทศไทย
\end{itemize}
ทั้งนี้ทั้งหมดหมดที่กล่าวมาเป็นเพียงส่วนหนึ่งของสวัสดิการที่ได้รับจากบริษัท G-Able และยังมีสวัสดิการอื่น ๆ ที่ยังไม่ได้กล่าวถึง

\section{วัฒนธรรมองค์กร}
บริษัท จีเอเบิล จำกัด (มหาชน) ตระหนักถึงความสำคัญของการมีระบบการกำกับดูแลกิจการที่ดี ซึ่งเป็นสิ่งสำคัญที่จะช่วยส่งเสริมการดำเนินงานของบริษัทให้มีประสิทธิภาพ และมีการเจริญเติบโตอย่างยั่งยืน

\clearpage